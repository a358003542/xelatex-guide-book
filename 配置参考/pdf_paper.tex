% !TEX TS-program = xelatex

% 根据 https://www.overleaf.com/latex/templates/constructions/vqhvxmnfgcbh
% 改动而成

\documentclass[12pt,a4paper,twoside]{article}
\newlength{\textpt}
\setlength{\textpt}{12pt}

\usepackage[T1]{fontenc}
\usepackage{tabularx, colortbl}
\usepackage[margin=3cm]{geometry}
\usepackage{xcolor}
\definecolor{ochre}{RGB}{204, 119, 34}
\usepackage[colorlinks=true,linkcolor=black,citecolor=black,filecolor=magenta,urlcolor=ochre]{hyperref}

\usepackage[final,nopatch=footnote]{microtype}

\usepackage{
	enumitem,
	multicol,
	multirow,
	linguex,
	amssymb,
	microtype,
	stmaryrd,
	amsmath,
	booktabs,
	calculator,
	verbatim,
	titling,
	fontspec,
	changepage,
	float,
    multirow
}

% Paper license
\usepackage[
    type={CC},
    modifier={by},
    version={4.0},
]{doclicense}

% set bibliography style
\bibliographystyle{unified}

%================字體================%
%設置英文字體
\setmainfont[Mapping=tex-text]{DejaVu Serif}
\setsansfont[Mapping=tex-text]{DejaVu Sans}
\setmonofont[Mapping=tex-text]{DejaVu Sans Mono}

%中文環境
\RequirePackage[]{xeCJK}
\xeCJKsetup{PunctStyle=plain}
\setCJKmainfont[FallBack=DejaVu Serif, ItalicFont=KaiTi]{Source Han Serif CN}
\setCJKsansfont[FallBack=DejaVu Sans]{Source Han Sans CN}
\setCJKmonofont[FallBack=DejaVu Sans Mono]{KaiTi}


% packages:
\usepackage[]{titlesec}
\usepackage[hang]{footmisc}
\usepackage[nocenter]{qtree}
\usepackage[normalem]{ulem}
\usepackage{graphicx}
\usepackage{fancyhdr}

% 默认图片地址
\graphicspath{{figures/}}

\setlength{\intextsep}{4pt plus 2pt minus 2pt}
\renewcommand{\firstrefdash}{}
% 中文行间距还是需要稍微拉大点
% 格式略微调整
\linespread{1.3}
\setlength{\parindent}{2em}
\RequirePackage{indentfirst} 
\setlength{\parskip}{1.6ex}

\newcommand{\ii}{\={i}}
\newcommand{\red}[1]{\textcolor{red}{#1}}
\newcommand{\Reference}{\\ \textcolor{white}{.}\hfill}

% header style for all but first page
\pagestyle{fancy}
\fancyhf{}
\renewcommand{\headrulewidth}{0pt}

% header style for first page
\fancypagestyle{plain}{%
  \fancyhf{}% clear all header and footer fields
  \renewcommand{\footrulewidth}{0.4pt}
  \lfoot{ }
  \rfoot{\thepage}%
}


% set head height
\setlength{\headheight}{14.49998pt}

\titleformat*{\section}{\large\bfseries}
\titleformat*{\subsection}{\normalsize\bfseries}
\titlespacing*{\section}{0em}{15pt}{10pt}


% for vertical centering text in tabularx X column
\renewcommand\tabularxcolumn[1]{m{#1}}

% column types for tabularx
\newcolumntype{L}{>{\raggedright\arraybackslash}X}
\newcolumntype{R}{>{\raggedleft\arraybackslash}X}
\newcolumntype{C}{>{\centering\arraybackslash}X}

\usepackage{caption}
\captionsetup{font=footnotesize}


% no vertical space in lists
\setlist{nosep}

% remove vertical space after verbatim environment
\usepackage{etoolbox}
\makeatletter
\preto{\@verbatim}{\topsep=0pt \partopsep=0pt }
\makeatother

% enumitem package seems to override this
% behavior, so we also add
% (thanks to https://tex.stackexchange.com/questions/43331/control-vertical-space-before-and-after-verbatim-environment)
\AtBeginEnvironment{verbatim}{\setlist[trivlist]{nolistsep}}


\fancyhead[LO]{}
\fancyhead[RE]{}
\fancyhead[RO, LE]{\thepage}


\raggedbottom
\begin{document}


% make sure that first page doesn't
% have running heads:
\thispagestyle{plain}

% first page header:
% DOI and Creative Commons license
\newcolumntype{Y}{>{\raggedleft\arraybackslash}X}

% 此处输入doi 一些文献管理软件比如zotero会自动解析出很多信息
\noindent
\begin{tabularx}{\textwidth}{XY}
\doclicenseImage[imagewidth=0.5\linewidth] & \small{doi{10.1093/mind/LIX.236.433}} \\
\end{tabularx}

\vspace{30pt}

% 标题
\begin{center}
    \LARGE{\textbf{计算机器与智能}}
\end{center}

% 作者信息
\begin{center}
\vspace{4pt}
\large
    Alan M. Turing
\end{center}


% 摘要
\begin{small}
\begin{center}
\vspace{9pt}
\textbf{Abstract}    
\end{center}

\begin{adjustwidth}{20pt}{20pt}
1950年,后来被誉为“计算机科学之父”和“人工智能先驱”的艾伦·图灵(Alan M. Turing)发表了划时代论文《计算机器与智能》(Computing Machinery and Intelligence),探讨“机器是否能思考”这一根本问题。文中图灵设计了“模仿游戏”(Imitation Game)的思想实验,这就是后世广为人知的“图灵测试”。然而,透过“图灵测试”的光环仔细研读,会发现图灵在文中表达了对机器与智能远远更为深刻的思考。为此集智俱乐部AGI社区【何瑞杰、徐博文、刘兆庭、冯睿洋| 翻译】对这篇经典论文做了完整翻译,希望能够回归本源,探讨其丰厚内涵,也希望对读者有所启发。
\end{adjustwidth}


\end{small}


\vspace{10pt}
\section{模仿游戏}
考虑以下问题:“机器能思考吗?”(Can machines think?)要回答这个问题,首先需要定义“机器”和“思考”这两个术语。定义的表述或许要尽可能反映这两个词的通常用法,但这种看法是危险的。如果我们通过考察这两个词的通常用法来找到它们的含义,那么很难避免得出这样的结论:这两个词的含义和“机器能思考吗”这一问题的答案,将在“盖洛普民意测验”之类的统计调查中寻求。这是荒谬的。我并不试图给出这样的定义,而是提出另一个问题,它与原问题密切相关,并用相对清晰的词汇表达。

新的问题可以通过一个游戏来描述,称之为“模仿游戏”(imitation game)。游戏中有一个男人(A)、一个女人(B)和一个提问者(C,不论男女)。提问者待在一个与另外两人相隔离的屋子里,其目标是判断出外面哪个是男人、哪个是女人。提问者用标签 X、Y 指称外面的两个人。游戏结束时,他要说出“X 是 A,Y 是 B”或者“X 是 B,Y 是 A”。例如,提问者 C 可以向 A 和 B 提出下面这样的问题:





\begin{thebibliography}{99}
\bibitem{1} Samuel Butler, Erewhon, London, 1865. Chapters 23, 24, 25, The Book of the Machines.
\bibitem{2} Alonzo Church, An Unsolvable Problem of Elementary Number Theory”, American J. of Math., 58(1936), 345–363.
\bibitem{3} K. Gödel, Über formal unentscheildbare Sätze der Principia Mathematica und verwandter Sys- teme, I”, Monatshefle für Math. und Phys. , (1931), 173–189.
\bibitem{4} D. R. Hartree, Calculating Instruments and Machines, New York, 1949.
\bibitem{5} S. C. Kleene, General Recursive Functions of Natural Numbers ”, American J. of Math., 57(1935), 153–173 and 219–244.
\bibitem{6} G. Jefferson, The Mind of Mechanical Man”. Lister Oration for 1949. British Medical Journal, vol.i (1949), 1105–1121.
\bibitem{7} Countess of Lovelace, Translator’s notes te an article on Babbage’s Analytical Engire”, Scientific Memoirs (ed. by R. Taylor), vol.3 (1842), 691–731.
\bibitem{8} Bertrand Russell, History of Western Philosophy, London, 1940.
\bibitem{9} A. M. Turing, On Computable Numbers, with an Application to the Entscheidungsproblem”,Proc.London Math.Soc. (2), 42 (1937), 230–265. Victoria University of Manchester.

\end{thebibliography}

\end{document}